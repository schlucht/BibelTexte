\documentclass[a4paper, 12pt]{article}

\usepackage[polutonikogreek, ngerman]{babel}
\usepackage[utf8x]{inputenc}
\usepackage[T1]{fontenc}
\usepackage{lmodern}
\usepackage{bibleref-german}
\usepackage{blindtext}
\usepackage{graphicx}
\usepackage{geometry}


%%%%%%%%%%%%%%%%%%%%%%Link Formatierung%%%%%%%%%%%%%%%%%%%%%%%%%%%
\usepackage[colorlinks = true,
linkcolor = black,
urlcolor  = blue,
citecolor = black,
anchorcolor = blue]{hyperref}

%%%%%%%%%%%%%%%%%%%%%Griechische Zeichen%%%%%%%%%%%%%%%%%%%%%%%%%%%%%%%%%%%
%\begin{otherlanguage}{polutonikogreek} 
%	Δοκεῖ δέ μοι καὶ Καρχηδόνα μὴ εἶναι. 
%\end{otherlanguage} 



%%%%%%%%%%%%%%%%%%%%%%%%%%%%%%%%%%%%%%%%%%%%%%%%%%%%%%%%%%%%%%%%%%
%
% Makro zur Namensvervollstädigung
% Parameter: Kürzel --- Bibel
%			LuN			Neue Luther Übersetzung
%			Sch2		Schlachter 2000
%			Elb			Elbfelder
%%%%%%%%%%%%%%%%%%%%%%%%%%%%%%%%%%%%%%%%%%%%%%%%%%%%%%%%%%%%%%%%%%%
	\newcommand{\bib}[1]{%
	\ifthenelse{\equal{#1}{EI}}{Einheitsübersetzung}{%
		\ifthenelse{\equal{#1}{Sch2}}{Schlachter 2000}{%
			\ifthenelse{\equal{#1}{HFA}}{Hoffnung für Alle}{%
				\ifthenelse{\equal{#1}{ELB}}{Elbfelder}{%
					\ifthenelse{\equal{#1}{Gr}}{Griechisch}{%
						\ifthenelse{\equal{#1}{LuN}}{Luther 2017}{#1}%
					}
				}%
			}%
		}%
	}
}
%%%%%%%%%%%%%%%%%%%%%%%%%%%%%%%%%%%%%%%%%%%%%%%%%%%%%%%%%%%%%%%%%%
%
% Makro für Bibelzitate
% 
% Beispiel: 
%		\begin{bibeltext}{ELB}{Matt}{1:1-4}
%			Ich bin der zitierte Bibeltext.
%		\end{bibeltext}
%
%%%%%%%%%%%%%%%%%%%%%%%%%%%%%%%%%%%%%%%%%%%%%%%%%%%%%%%%%%%%%%%%%%
\newcommand{\bibtit}[1]{\large\bib{#1}}
\newcommand{\tmpbib}{Hallo}
\newenvironment{bibeltext}[3]{%
	\quote \begin{itshape}
		\begin{scriptsize}
			(\bib{#1})	
		\end{scriptsize}	
		\biblerefformat{lang}
		\renewcommand{\tmpbib}{\textbf{\bibleverse{#2}(#3)}}		
	}{%	
		\tmpbib
	\end{itshape}
	\endquote		
}
%%%%%%%%%%%%% Titelseite %%%%%%%%%%%%%%%%%%%%%%%%%%%%%%%%%%%%%%%%%%%%%
\title{Freikirchen, Katholische Kirchen, Evangelische Kirchen}
\author{Schmid Lothar}
\date{\today}
\pagestyle{headings}

%%%%%%%%%%%%%%%%% Beginn Dokument %%%%%%%%%%%%%%%%%%%%%%%%%%%%%%%%%%%%


\begin{document}

\maketitle[-1]
\section{Vorwort}

Wenn sich ein bisschen mit den Kirchen, Religionen und Glauben beschäftigt, kommt man nicht um den Begriff Ökumene vorbei. Es gibt Kirchen die Ökumene als die Zukunft und das einzige heilbringende darstellen und andere wiederum die Ökumene verteufeln.\\
Ich weiss nicht ob es wirklich gut ist oder nicht. Darum versuche ich in diesem Schreiben meine Gedanken zu Papier zubringen und schauen was denn die Bibel dazu sagt. In erster Linie tönt es gut. Gemeinschaft mit anderen, anderen Gemeinschaften Respekt entgegen bringen...\\
Ist das wirklich so? Welche Religion will jetzt welche aufnehmen? Geht es um religiöse Fusionen oder um Mitgliederbewerbung. \\ 
Die Welt des Internet ist voll davon. Auf YouTube gibt es hunderte von Predigten welch sich um Ökumene befassen. Ich bin weder Psychologe noch Theologe sondern einfach nur ein Gläubiger Christ den das ganze interessiert und gerne seine Gedanken zu Papier bringen möchte.\\
Ich möchte das gerne mit dem Geist Gottes zusammen machen. So dass er mich leitet und mich unterstützt.\\\\

In dem Sinne werde ich meine Gedanken zu Papier bringen.








\part{Bibel}
\section {Der Philipperbrief}
Der Brief an die Gemeinde Philipi ist ein kurzer Brief von Paulus. Er enthält gerade mal 4 Kapitel. Paulus schreibt diesen Brief aus dem Gefängnis in Rom.

\begin{bibeltext}{Gr}{Phil}{1:13}
\textsuperscript{13}
\textgreek{ 	
	ὥστε τοὺς δεσμούς μου φανεροὺς ἐν Χριστῷ γενέσθαι ἐν ὅλῳ τῷ πραιτωρίῳ καὶ τοῖς λοιποῖς πάσιν, }
\end{bibeltext}

\begin{bibeltext}{Sch2}{Phil}{1:13}
	\textsuperscript{13}so dass in der ganzen kaiserlichen Kaserne und bei allen Übrigen bekannt geworden ist, dass ich um des Christus willen gefesselt bin.
\end{bibeltext}
\begin{bibeltext}{HFA}{Phil}{1:13}
	\textsuperscript{13}Allen meinen Bewachern und auch den übrigen Menschen, mit denen ich es zu tun habe, ist inzwischen klar geworden, dass ich nur deswegen eingesperrt bin, weil ich an Christus glaube.
\end{bibeltext}
\begin{bibeltext}{ELB}{Phil}{1:13}
	\textsuperscript{13}so dass meine Fesseln in Christus im ganzen Prätorium und bei allen anderen offenbar geworden sind 
\end{bibeltext}
\begin{bibeltext}{EI}{Phil}{1:13}
	\textsuperscript{13}Denn im ganzen Prätorium und bei allen Übrigen ist offenbar geworden, dass ich meine Fesseln um Christi willen trage,
\end{bibeltext}

Es beweist also, dass Paulus beim verfassen oder diktieren dieses Briefes in einem Gefängnis saß. Da zu der Zeit der Kaiser in Rom seinen Sitz hatte, liegt es nahe, dass der Brief auch in Rom geschrieben wurde.

\subsection{Kapitel 1}

\part{Allgemein}
\section{Das Paradies}
\subsection {Was die Bibel sagt}
\begin{bibeltext}{LuN}{1Mos}{2:15}
	\textsuperscript{15} Und Gott der HERR nahm den Menschen und setzte ihn in den Garten Eden, dass er ihn bebaute und bewahrte.
\end{bibeltext}

\begin{bibeltext}{Sch2}{1Mos}{2:15}
	\textsuperscript{15} Und Gott der HERR nahm den Menschen und setzte ihn in den Garten Eden, damit er ihn bebaue und bewahre
\end{bibeltext}

Hier in diesem Vers kommt deutlich hervor, dass im Paradies nicht nur Harfe gespielt und gesungen wird. Das Paradies wurde als Landwirtschafts Gebiet geschaffen. Also nichts mit gemütlich hinlegen und nichts tun.\\
Wir werden Gärtner im Himmel und erhalten ein Häuschen und ein Stück Land. Vielleicht auch noch einen Löwen als Hauskatze.\\
Schon in diesem Vers zeigt, das Gott uns nicht geschaffen um im Bett zu liegen und umher zu lungern. \flqq damit er ihn bebaue\frqq{} Adam und Eva waren also Gärtner im Paradies. Das Bebauen des Gartens war aber im Paradies nicht müselig, sondern eine angenehme Arbeit. Bis zum Sündenfall gab es kein Unkraut und harte Böden.
\begin{bibeltext}{Sch2}{1Mos}{2:10}
    \textsuperscript{15}Es ging aber ein Strom aus von Eden, um den Garten zu bewässern; von dort aber teilte er sich und wurde zu vier Hauptströmen.
\end{bibeltext}
Wie wir hier lesen, war auch eine Bewässerungsanlage installiert. Also eigentlich alles was man braucht. Regen gab es zu der Zeit noch keinen. Das lesen wir wir in
\begin{bibeltext}{Sch2}{1Mos}{9:13-14}
    \textsuperscript{13}Meinen Bogen setze ich in die Wolken, der soll ein Zeichen des Bundes sein zwischen mir und der Erde.
    \textsuperscript{14} Wenn es nun geschieht, dass ich Wolken über der Erde sammle, und der Bogen in den Wolken erscheint,
\end{bibeltext}
Das war nach der Sindflut. Gott hat als Zeichen des Bundes einen Regenbogen genommen. Er setzt erst hier den Bogen in die Wolken. Vielleicht gab es vor der Sindflut keinen Regen. Warum sollte Gott ein Zeichen für einen Bund nehmen, dass schon immer im Himmel stand?
Darum erschien auch kein Regenbogen. Oder die Regentropfen hatten eine andere Form und konnten so das Licht nicht brechen. Jedenfalls lesen wir in der Bibel das erste mal bei der Sindflut von Regen.\\
Wie dem auch sei, die Gärtner im Paradies hatten zwar Arbeit, aber die Arbeit hat spass gemacht und war nicht mühselig. Wir Menschen müssen ja immer etwas tun. Was wäre das für ein Paradies, wenn man nur umher liegt und sich langweilt? Zu Tode langweilen kann man sich ja da nicht mehr.\\ Ist die neue Welt gleich wie das Paradies? Haben wir da Garten und ein Häuschen?\\
\begin{bibeltext}{Sch2}{Jes}{11:6-7}
    \textsuperscript{6}Da wird der Wolf bei dem Lämmlein wohnen und der Leopard sich bei dem Böcklein niederlegen. Das Kalb, der junge Löwe und das Mastvieh werden beieinander sein, und ein kleiner Knabe wird sie treiben.
    \textsuperscript{7} Die Kuh und die Bärin werden miteinander weiden und ihre Jungen zusammen lagern, und der Löwe wird Stroh fressen wie das Rind.
\end{bibeltext}
Hier in Jesaia wird beschrieben wie es einmal aussehen wird. Also es gibt Stroh und Gras. Also Wiesen und Äcker.


\end{document}
