\documentclass[a4paper, 11pt]{article}


\usepackage[ngerman]{babel}
\usepackage[utf8]{inputenc}
\usepackage[T1]{fontenc}
\usepackage{lmodern}
\usepackage{bibleref-german}
\usepackage{blindtext}
\usepackage{graphicx}
\usepackage{geometry}


%%%%%%%%%%%%%%%%%%%%%%Link Formatierung%%%%%%%%%%%%%%%%%%%%%%%%%%%
\usepackage[colorlinks = true,
linkcolor = black,
urlcolor  = blue,
citecolor = black,
anchorcolor = blue]{hyperref}

%%%%%%%%%%%%%%%%%%%%%Griechische Zeichen%%%%%%%%%%%%%%%%%%%%%%%%%%%%%%%%%%%
%\begin{otherlanguage}{polutonikogreek} 
%	Δοκεῖ δέ μοι καὶ Καρχηδόνα μὴ εἶναι. 
%\end{otherlanguage} 



%%%%%%%%%%%%%%%%%%%%%%%%%%%%%%%%%%%%%%%%%%%%%%%%%%%%%%%%%%%%%%%%%%
%
% Makro zur Namensvervollstädigung
% Parameter: Kürzel --- Bibel
%			LuN			Neue Luther Übersetzung
%			Sch2		Schlachter 2000
%			Elb			Elbfelder
%%%%%%%%%%%%%%%%%%%%%%%%%%%%%%%%%%%%%%%%%%%%%%%%%%%%%%%%%%%%%%%%%%%
	\newcommand{\bib}[1]{%
	\ifthenelse{\equal{#1}{EI}}{Einheitsübersetzung}{%
		\ifthenelse{\equal{#1}{Sch2}}{Schlachter 2000}{%
			\ifthenelse{\equal{#1}{HFA}}{Hoffnung für Alle}{%
				\ifthenelse{\equal{#1}{ELB}}{Elbfelder}{%
					\ifthenelse{\equal{#1}{Gr}}{Griechisch}{%
						\ifthenelse{\equal{#1}{LuN}}{Luther 2017}{#1}%
					}
				}%
			}%
		}%
	}
}
%%%%%%%%%%%%%%%%%%%%%%%%%%%%%%%%%%%%%%%%%%%%%%%%%%%%%%%%%%%%%%%%%%
%
% Makro für Bibelzitate
% 
% Beispiel: 
%		\begin{bibeltext}{ELB}{Matt}{1:1-4}
%			Ich bin der zitierte Bibeltext.
%		\end{bibeltext}
%
%%%%%%%%%%%%%%%%%%%%%%%%%%%%%%%%%%%%%%%%%%%%%%%%%%%%%%%%%%%%%%%%%%
\newcommand{\bibtit}[1]{\large\bib{#1}}
\newcommand{\tmpbib}{Hallo}
\newenvironment{bibeltext}[3]{%
	\quote \begin{itshape}
		\begin{scriptsize}
			(\bib{#1})	
		\end{scriptsize}	
		\biblerefformat{lang}
		\renewcommand{\tmpbib}{\textbf{\bibleverse{#2}(#3)}}		
	}{%	
		\tmpbib
	\end{itshape}
	\endquote		
}
%%%%%%%%%%%%% Titelseite %%%%%%%%%%%%%%%%%%%%%%%%%%%%%%%%%%%%%%%%%%%%%
\title{Bibelschule MSCN}
\author{Schmid Lothar}
\date{\today}
\pagestyle{headings}

%%%%%%%%%%%%%%%%% Beginn Dokument %%%%%%%%%%%%%%%%%%%%%%%%%%%%%%%%%%%%


\begin{document}
\tableofcontents

\newpage
\section{Vorwort}
\section{Allgemein}
Janina und ich haben uns entschieden einen Bibelkurs bei der Freien Evangelischen Kirche Mitternachtsruf durch zu führen. Durch diesen Kurs erhoffen wir, dass wir eine bessere Beziehung zu Jesus aufbauen können.

\subsection{Mitternachtsruf}
Der Mitternachtsruf \footnote{www.mnr.ch} ist ein freies evangelisches Missionswerk mit dem Ziel, die Menschen auf Jesus Christus, Seine frohe Botschaft und Seine Rückkehr auf diese Erde hinzuweisen. Ihr Name leitet sich ab \begin{bibeltext}{Sch2}{Matt}{25:6}
    			Um Mitternacht aber entstand ein Geschrei: \glqq Siehe der Bräutigam kommt! Geht aus, ihm entgegen!\grqq{}
    		\end{bibeltext}
Diese Gemeinde bietet eine Bibelschule\footnote{Gemeinde Bibel Schule Mitternachtsruf (GBSM)} an, die gratis ist und 2 Jahre dauert. 

\subsection{Themen}
\begin{itemize}
    \item \textbf{Aionologie: }die Lehre von den Zeitaltern. was sagt die Bibel über die verschiedenen Zeitalter in Gottes Heilsplan?
    \item \textbf{Anthropologie: }die Lehre von dem Menschen.
    Was sagt die Bibel über Ursprung, Ziel und Wesen des Menschen?
    \item \textbf{Bibiologie: }die Lehre des Buches. Wie ist die Bibel entstanden und was sagt sie über sich selbst aus?
    \item \textbf{Biblische Geografie: }die Kunde des Landes der Bibel. Wo befinden sich und welche Rolle spielen die historischen Orte der Bibel?
    \item \textbf{Ekkesiologie: }die Lehre von der Gemeinde. Was sagt die Bibel über die Gemeinde des lebendigen Gottes?
    \item \textbf{Eschatologie: }die Lehre von den letzten Dingen. Was sagt die Bibel über die Zukunft des Menschen und der Welt?
    \item \textbf{Hamartiologie: }Hamartiologie: die Lehre von der Sünde. Was sagt die Bibel über das Wesen und die
    Auswirkungen der Sünde?
    \item \textbf{Israelogie: }die Lehre von Israel. Was sagt die Bibel über das Volk Israel?
    \item \textbf{Missionologie: }die Lehre von der Mission. Was sagt die Bibel über den Missionsauftrag der Christen?
    \item \textbf{Theologie: }die Lehre von Gott. Was sagt die Bibel über Gott selbst und wie hat sich diese Lehre in der Gemeinde entwickelt?
\end{itemize}
\subsection{Autoren}
\begin{itemize}
    \item Norbert Lieth
    \item Thomas Lieth
    \item Fredy Peter
    \item Nathanael Winkler
    \item Samuel Rindlisbacher
    \item René Malgo
\end{itemize}
\subsection{Ort}
\parbox{3.5in}{\textbf{Maranatha Haus}}\\
\parbox{3.5in}{Zionsweg 1} \\
\parbox{3.5in}{8600 Dübendorf} \\
\parbox{3.5in}{Webseite: www.gbsm.ch}
\newpage
\section{Kurs}
\subsection{31.1.2021 Erster Studien Tag in Dübendorf}
\subsubsection{Aufgaben}
\begin{itemize}
    \item \textbf{Biblelesen:} 1. Moses die Kapitel 1 - 50
    \item \textbf{Bibiologie:} Kapitel 1 -4 im Buch "Grundlagen biblischer Lehre" 
    \cite{bibli:1} und zu jedem Kapitel mindestens die ersten drei Fragen beantworten. 
\end{itemize}
\section{Kurs}
\subsection{31.1.2021 Erster Studien Tag in Dübendorf}
\subsubsection{Aufgaben}
\begin{itemize}
    \item \textbf{Biblelesen:} 1. Moses die Kapitel 1 - 50
    \item \textbf{Bibiologie:} Kapitel 1 -4 im Buch "Grundlagen biblischer Lehre" \cite{biblischelehre} und zu jedem Kapitel mindestens die ersten drei Fragen beanworten. 
\end{itemize}


%\section {Der Philipperbrief}
Der Brief an die Gemeinde Philipi ist ein kurzer Brief von Paulus. Er enthält gerade mal 4 Kapitel. Paulus schreibt diesen Brief aus dem Gefängnis in Rom.

\begin{bibeltext}{Gr}{Phil}{1:13}
\textsuperscript{13}
\textgreek{ 	
	ὥστε τοὺς δεσμούς μου φανεροὺς ἐν Χριστῷ γενέσθαι ἐν ὅλῳ τῷ πραιτωρίῳ καὶ τοῖς λοιποῖς πάσιν, }
\end{bibeltext}

\begin{bibeltext}{Sch2}{Phil}{1:13}
	\textsuperscript{13}so dass in der ganzen kaiserlichen Kaserne und bei allen Übrigen bekannt geworden ist, dass ich um des Christus willen gefesselt bin.
\end{bibeltext}
\begin{bibeltext}{HFA}{Phil}{1:13}
	\textsuperscript{13}Allen meinen Bewachern und auch den übrigen Menschen, mit denen ich es zu tun habe, ist inzwischen klar geworden, dass ich nur deswegen eingesperrt bin, weil ich an Christus glaube.
\end{bibeltext}
\begin{bibeltext}{ELB}{Phil}{1:13}
	\textsuperscript{13}so dass meine Fesseln in Christus im ganzen Prätorium und bei allen anderen offenbar geworden sind 
\end{bibeltext}
\begin{bibeltext}{EI}{Phil}{1:13}
	\textsuperscript{13}Denn im ganzen Prätorium und bei allen Übrigen ist offenbar geworden, dass ich meine Fesseln um Christi willen trage,
\end{bibeltext}

Es beweist also, dass Paulus beim verfassen oder diktieren dieses Briefes in einem Gefängnis saß. Da zu der Zeit der Kaiser in Rom seinen Sitz hatte, liegt es nahe, dass der Brief auch in Rom geschrieben wurde.

\subsection{Kapitel 1}




\end{document}
