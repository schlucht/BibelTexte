\documentclass[a4paper, 12pt]{article}

\usepackage[ngerman]{babel}
\usepackage[utf8]{inputenc}
\usepackage[T1]{fontenc}
\usepackage{lmodern}
\usepackage{bibleref-german}
\usepackage{blindtext}
\usepackage{graphicx}
\usepackage{geometry}


%%%%%%%%%%%%%%%%%%%%%%Link Formatierung%%%%%%%%%%%%%%%%%%%%%%%%%%%
\usepackage[colorlinks = true,
linkcolor = black,
urlcolor  = blue,
citecolor = black,
anchorcolor = blue]{hyperref}

%%%%%%%%%%%%%%%%%%%%%Griechische Zeichen%%%%%%%%%%%%%%%%%%%%%%%%%%%%%%%%%%%
%\begin{otherlanguage}{polutonikogreek} 
%	Δοκεῖ δέ μοι καὶ Καρχηδόνα μὴ εἶναι. 
%\end{otherlanguage} 



%%%%%%%%%%%%%%%%%%%%%%%%%%%%%%%%%%%%%%%%%%%%%%%%%%%%%%%%%%%%%%%%%%
%
% Makro zur Namensvervollstädigung
% Parameter: Kürzel --- Bibel
%			LuN			Neue Luther Übersetzung
%			Sch2		Schlachter 2000
%			Elb			Elbfelder
%%%%%%%%%%%%%%%%%%%%%%%%%%%%%%%%%%%%%%%%%%%%%%%%%%%%%%%%%%%%%%%%%%%
	\newcommand{\bib}[1]{%
	\ifthenelse{\equal{#1}{EI}}{Einheitsübersetzung}{%
		\ifthenelse{\equal{#1}{Sch2}}{Schlachter 2000}{%
			\ifthenelse{\equal{#1}{HFA}}{Hoffnung für Alle}{%
				\ifthenelse{\equal{#1}{ELB}}{Elbfelder}{%
					\ifthenelse{\equal{#1}{Gr}}{Griechisch}{%
						\ifthenelse{\equal{#1}{LuN}}{Luther 2017}{#1}%
					}
				}%
			}%
		}%
	}
}
%%%%%%%%%%%%%%%%%%%%%%%%%%%%%%%%%%%%%%%%%%%%%%%%%%%%%%%%%%%%%%%%%%
%
% Makro für Bibelzitate
% 
% Beispiel: 
%		\begin{bibeltext}{ELB}{Matt}{1:1-4}
%			Ich bin der zitierte Bibeltext.
%		\end{bibeltext}
%
%%%%%%%%%%%%%%%%%%%%%%%%%%%%%%%%%%%%%%%%%%%%%%%%%%%%%%%%%%%%%%%%%%
\newcommand{\bibtit}[1]{\large\bib{#1}}
\newcommand{\tmpbib}{Hallo}
\newenvironment{bibeltext}[3]{%
	\quote \begin{itshape}
		\begin{scriptsize}
			(\bib{#1})	
		\end{scriptsize}	
		\biblerefformat{lang}
		\renewcommand{\tmpbib}{\textbf{\bibleverse{#2}(#3)}}		
	}{%	
		\tmpbib
	\end{itshape}
	\endquote		
}
%%%%%%%%%%%%% Titelseite %%%%%%%%%%%%%%%%%%%%%%%%%%%%%%%%%%%%%%%%%%%%%
\title{Bibelschule MSCN}
\author{Schmid Lothar}
\date{\today}
\pagestyle{headings}

%%%%%%%%%%%%%%%%% Beginn Dokument %%%%%%%%%%%%%%%%%%%%%%%%%%%%%%%%%%%%


\begin{document}

\maketitle[-1]

\section{Vorwort}

Wenn sich ein bisschen mit den Kirchen, Religionen und Glauben beschäftigt, kommt man nicht um den Begriff Ökumene vorbei. Es gibt Kirchen die Ökumene als die Zukunft und das einzige heilbringende darstellen und andere wiederum die Ökumene verteufeln.\\
Ich weiss nicht ob es wirklich gut ist oder nicht. Darum versuche ich in diesem Schreiben meine Gedanken zu Papier zubringen und schauen was denn die Bibel dazu sagt. In erster Linie tönt es gut. Gemeinschaft mit anderen, anderen Gemeinschaften Respekt entgegen bringen...\\
Ist das wirklich so? Welche Religion will jetzt welche aufnehmen? Geht es um religiöse Fusionen oder um Mitgliederbewerbung. \\ 
Die Welt des Internet ist voll davon. Auf YouTube gibt es hunderte von Predigten welch sich um Ökumene befassen. Ich bin weder Psychologe noch Theologe sondern einfach nur ein Gläubiger Christ den das ganze interessiert und gerne seine Gedanken zu Papier bringen möchte.\\
Ich möchte das gerne mit dem Geist Gottes zusammen machen. So dass er mich leitet und mich unterstützt.\\\\

In dem Sinne werde ich meine Gedanken zu Papier bringen.









/section{Grundlegendes zur Bibel}
Hier wird das Grundlegende in der Bibel bearbeitet
/subsection{Kapitel 1}

%\section {Der Philipperbrief}
Der Brief an die Gemeinde Philipi ist ein kurzer Brief von Paulus. Er enthält gerade mal 4 Kapitel. Paulus schreibt diesen Brief aus dem Gefängnis in Rom.

\begin{bibeltext}{Gr}{Phil}{1:13}
\textsuperscript{13}
\textgreek{ 	
	ὥστε τοὺς δεσμούς μου φανεροὺς ἐν Χριστῷ γενέσθαι ἐν ὅλῳ τῷ πραιτωρίῳ καὶ τοῖς λοιποῖς πάσιν, }
\end{bibeltext}

\begin{bibeltext}{Sch2}{Phil}{1:13}
	\textsuperscript{13}so dass in der ganzen kaiserlichen Kaserne und bei allen Übrigen bekannt geworden ist, dass ich um des Christus willen gefesselt bin.
\end{bibeltext}
\begin{bibeltext}{HFA}{Phil}{1:13}
	\textsuperscript{13}Allen meinen Bewachern und auch den übrigen Menschen, mit denen ich es zu tun habe, ist inzwischen klar geworden, dass ich nur deswegen eingesperrt bin, weil ich an Christus glaube.
\end{bibeltext}
\begin{bibeltext}{ELB}{Phil}{1:13}
	\textsuperscript{13}so dass meine Fesseln in Christus im ganzen Prätorium und bei allen anderen offenbar geworden sind 
\end{bibeltext}
\begin{bibeltext}{EI}{Phil}{1:13}
	\textsuperscript{13}Denn im ganzen Prätorium und bei allen Übrigen ist offenbar geworden, dass ich meine Fesseln um Christi willen trage,
\end{bibeltext}

Es beweist also, dass Paulus beim verfassen oder diktieren dieses Briefes in einem Gefängnis saß. Da zu der Zeit der Kaiser in Rom seinen Sitz hatte, liegt es nahe, dass der Brief auch in Rom geschrieben wurde.

\subsection{Kapitel 1}




\end{document}
