\section {Der Philipperbrief}
Der Brief an die Gemeinde Philipi ist ein kurzer Brief von Paulus. Er enthält gerade mal 4 Kapitel. Paulus schreibt diesen Brief aus dem Gefängnis in Rom.

\begin{bibeltext}{Gr}{Phil}{1:13}
\textsuperscript{13}
\textgreek{ 	
	ὥστε τοὺς δεσμούς μου φανεροὺς ἐν Χριστῷ γενέσθαι ἐν ὅλῳ τῷ πραιτωρίῳ καὶ τοῖς λοιποῖς πάσιν, }
\end{bibeltext}

\begin{bibeltext}{Sch2}{Phil}{1:13}
	\textsuperscript{13}so dass in der ganzen kaiserlichen Kaserne und bei allen Übrigen bekannt geworden ist, dass ich um des Christus willen gefesselt bin.
\end{bibeltext}
\begin{bibeltext}{HFA}{Phil}{1:13}
	\textsuperscript{13}Allen meinen Bewachern und auch den übrigen Menschen, mit denen ich es zu tun habe, ist inzwischen klar geworden, dass ich nur deswegen eingesperrt bin, weil ich an Christus glaube.
\end{bibeltext}
\begin{bibeltext}{ELB}{Phil}{1:13}
	\textsuperscript{13}so dass meine Fesseln in Christus im ganzen Prätorium und bei allen anderen offenbar geworden sind 
\end{bibeltext}
\begin{bibeltext}{EI}{Phil}{1:13}
	\textsuperscript{13}Denn im ganzen Prätorium und bei allen Übrigen ist offenbar geworden, dass ich meine Fesseln um Christi willen trage,
\end{bibeltext}

Es beweist also, dass Paulus beim verfassen oder diktieren dieses Briefes in einem Gefängnis saß. Da zu der Zeit der Kaiser in Rom seinen Sitz hatte, liegt es nahe, dass der Brief auch in Rom geschrieben wurde.

\subsection{Kapitel 1}
