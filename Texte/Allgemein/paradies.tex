\section{Das Paradies}
\subsection {Was die Bibel sagt}
\begin{bibeltext}{LuN}{1Mos}{2:15}
	\textsuperscript{15} Und Gott der HERR nahm den Menschen und setzte ihn in den Garten Eden, dass er ihn bebaute und bewahrte.
\end{bibeltext}

\begin{bibeltext}{Sch2}{1Mos}{2:15}
	\textsuperscript{15} Und Gott der HERR nahm den Menschen und setzte ihn in den Garten Eden, damit er ihn bebaue und bewahre
\end{bibeltext}

Hier in diesem Vers kommt deutlich hervor, dass im Paradies nicht nur Harfe gespielt und gesungen wird. Das Paradies wurde als Landwirtschafts Gebiet geschaffen. Also nichts mit gemütlich hinlegen und nichts tun.\\
Wir werden Gärtner im Himmel und erhalten ein Häuschen und ein Stück Land. Vielleicht auch noch einen Löwen als Hauskatze.\\
Schon in diesem Vers zeigt, das Gott uns nicht geschaffen um im Bett zu liegen und umher zu lungern. "damit er ihn bebaue" Adam und Eva waren also Gärtner im Paradies. Das Bebauen des Gartens war aber im Paradies nicht müselig, sondern eine angenehme Arbeit. Bis zum Sündenfall gab es kein Unkraut und harte Böden.
\begin{bibeltext}{Sch2}{1Mos}{2:10}
    \textsuperscript{15}Es ging aber ein Strom aus von Eden, um den Garten zu bewässern; von dort aber teilte er sich und wurde zu vier Hauptströmen.
\end{bibeltext}
Wie wir hier lesen, war auch eine Bewässerungsanlage installiert. Also eigentlich alles was man braucht. Regen gab es zu der Zeit noch keinen. Das lesen wir wir in
\begin{bibeltext}{Sch2}{1Mos}{9:13-14}
    \textsuperscript{13}Meinen Bogen setze ich in die Wolken, der soll ein Zeichen des Bundes sein zwischen mir und der Erde.
    \textsuperscript{14} Wenn es nun geschieht, dass ich Wolken über der Erde sammle, und der Bogen in den Wolken erscheint,
\end{bibeltext}
Das war nach der Sindflut. Gott als Zeichen des Bundes einen Regenbogen genommen. Er setzt erst hier den Bogen in die Wolken. Vielleicht gab es vor der Sindflut keinen Regen. Darum erschien auch kein Regenbogen. Oder die Regentropfen hatten eine andere Form und konnten so das Licht nicht brechen. Jedenfalls lesen wir in der Bibel das erste mal bei der Sindflut von Regen gesprochen.
