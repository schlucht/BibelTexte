\chapter{Allgemein}
Janina und ich haben uns entschieden einen Bibelkurs bei der Freien Evangelischen Kirche Mitternachtsruf durch zu führen. Durch diesen Kurs erhoffen wir, dass wir eine bessere Beziehung zu Jesus aufbauen können.

\section{Mitternachtsruf}
Der Mitternachtsruf \footnote{www.mnr.ch} ist ein freies evangelisches Missionswerk mit dem Ziel, die Menschen auf Jesus Christus, Seine frohe Botschaft und Seine Rückkehr auf diese Erde hinzuweisen. Ihr Name leitet sich ab \begin{bibeltext}{Sch2}{Matt}{25:6}
    			Um Mitternacht aber entstand ein Geschrei: \glqq Siehe der Bräutigam kommt! Geht aus, ihm entgegen!\grqq{}
    		\end{bibeltext}
Diese Gemeinde bietet eine Bibelschule\footnote{Gemeinde Bibel Schule Mitternachtsruf (GBSM)} an, die gratis ist und 2 Jahre dauert. 

\section{Themen}
\begin{itemize}
    \item \textbf{Aionologie: }die Lehre von den Zeitaltern. was sagt die Bibel über die verschiedenen Zeitalter in Gottes Heilsplan?
    \item \textbf{Anthropologie: }die Lehre von dem Menschen.
    Was sagt die Bibel über Ursprung, Ziel und Wesen des Menschen?
    \item \textbf{Bibiologie: }die Lehre des Buches. Wie ist die Bibel entstanden und was sagt sie über sich selbst aus?
    \item \textbf{Biblische Geografie: }die Kunde des Landes der Bibel. Wo befinden sich und welche Rolle spielen die historischen Orte der Bibel?
    \item \textbf{Ekkesiologie: }die Lehre von der Gemeinde. Was sagt die Bibel über die Gemeinde des lebendigen Gottes?
    \item \textbf{Eschatologie: }die Lehre von den letzten Dingen. Was sagt die Bibel über die Zukunft des Menschen und der Welt?
    \item \textbf{Hamartiologie: }Hamartiologie: die Lehre von der Sünde. Was sagt die Bibel über das Wesen und die
    Auswirkungen der Sünde?
    \item \textbf{Israelogie: }die Lehre von Israel. Was sagt die Bibel über das Volk Israel?
    \item \textbf{Missionologie: }die Lehre von der Mission. Was sagt die Bibel über den Missionsauftrag der Christen?
    \item \textbf{Theologie: }die Lehre von Gott. Was sagt die Bibel über Gott selbst und wie hat sich diese Lehre in der Gemeinde entwickelt?
\end{itemize}
\section{Autoren}
\begin{itemize}
    \item Norbert Lieth
    \item Thomas Lieth
    \item Fredy Peter
    \item Nathanael Winkler
    \item Samuel Rindlisbacher
    \item Philip Ottenburg 
\end{itemize}
\section{Ort}
\parbox{3.5in}{\textbf{Maranatha Haus}}\\
\parbox{3.5in}{Zionsweg 1} \\
\parbox{3.5in}{8600 Dübendorf} \\
\parbox{3.5in}{Webseite: www.gbsm.ch}
\newpage
\section{Kurs}
\subsection{31.1.2021 Erster Studien Tag in Dübendorf}
\subsubsection{Aufgaben}
\begin{itemize}
    \item \textbf{Biblelesen:} 1. Moses die Kapitel 1 - 50
    \item \textbf{Bibiologie:} Kapitel 1 -4 im Buch "Grundlagen biblischer Lehre" 
    \cite{bibli:1} und zu jedem Kapitel mindestens die ersten drei Fragen beantworten. 
\end{itemize}
\subsubsection{Ablauf}
Wegen dem Corona Problem konnte der Kurs nur Online durchgeführt werden. Pünktlich um 12:45 Uhr hat der Kurs angefangen. An Anfang wurden alle Lehrer vorgestellt und jeder hatte von ihnen hatte ein Begrüssungswort gesprochen.

Nach ein paar organisatorischen Worten hat der Kurs mit dem Thema über die Bibel begonnen. Der erste Vortrag über die Bibel wurde von Fredy Peter gehalten. Der Referent wirkte sehr kompetent und man spürte seine Liebe zu Gott und zu seinem Buch. Es ist beeindruckend wie er sich in ein Thema rein steigern kann.

Die Lektionen sind in 45 Minuten und 15min Pause unterteilt. Nach dem ersten Teil der Bibiologie gab es dann eine Pause von 30 Minuten. Danach kam das erste Buch Mose dran. Das Buch wurde von Nathanel Winkler vorgestellt. Sehr spannend und interessant vorgetragen.

Natürlich konnte man in 2h nicht die ganzen 50 Kapitel im Detail anschauen. Aber es gab einen kurzen Überblick über die wichtigsten Themen im Buch. Es wurde nach den 5W--Fragen vorgegangen:
\begin{enumerate}
    \item WER --- Author?
    \item WEM --- Adressaten?
    \item WANN --- Zeit und Ort der Abfassung?
    \item WAS --- Inhalt, Anlass und Problemstellung
    \item WIE --- Grobstruktur, Aufbau, Stil
\end{enumerate}
\subsection{28.2.2021 Zweiter Studien Tag in Dübendorf}